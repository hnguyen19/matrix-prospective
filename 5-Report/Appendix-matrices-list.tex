% Options for packages loaded elsewhere
\PassOptionsToPackage{unicode}{hyperref}
\PassOptionsToPackage{hyphens}{url}
%
\documentclass[
]{article}
\usepackage{amsmath,amssymb}
\usepackage{lmodern}
\usepackage{iftex}
\ifPDFTeX
  \usepackage[T1]{fontenc}
  \usepackage[utf8]{inputenc}
  \usepackage{textcomp} % provide euro and other symbols
\else % if luatex or xetex
  \usepackage{unicode-math}
  \defaultfontfeatures{Scale=MatchLowercase}
  \defaultfontfeatures[\rmfamily]{Ligatures=TeX,Scale=1}
\fi
% Use upquote if available, for straight quotes in verbatim environments
\IfFileExists{upquote.sty}{\usepackage{upquote}}{}
\IfFileExists{microtype.sty}{% use microtype if available
  \usepackage[]{microtype}
  \UseMicrotypeSet[protrusion]{basicmath} % disable protrusion for tt fonts
}{}
\makeatletter
\@ifundefined{KOMAClassName}{% if non-KOMA class
  \IfFileExists{parskip.sty}{%
    \usepackage{parskip}
  }{% else
    \setlength{\parindent}{0pt}
    \setlength{\parskip}{6pt plus 2pt minus 1pt}}
}{% if KOMA class
  \KOMAoptions{parskip=half}}
\makeatother
\usepackage{xcolor}
\usepackage[margin=1in]{geometry}
\usepackage{longtable,booktabs,array}
\usepackage{calc} % for calculating minipage widths
% Correct order of tables after \paragraph or \subparagraph
\usepackage{etoolbox}
\makeatletter
\patchcmd\longtable{\par}{\if@noskipsec\mbox{}\fi\par}{}{}
\makeatother
% Allow footnotes in longtable head/foot
\IfFileExists{footnotehyper.sty}{\usepackage{footnotehyper}}{\usepackage{footnote}}
\makesavenoteenv{longtable}
\usepackage{graphicx}
\makeatletter
\def\maxwidth{\ifdim\Gin@nat@width>\linewidth\linewidth\else\Gin@nat@width\fi}
\def\maxheight{\ifdim\Gin@nat@height>\textheight\textheight\else\Gin@nat@height\fi}
\makeatother
% Scale images if necessary, so that they will not overflow the page
% margins by default, and it is still possible to overwrite the defaults
% using explicit options in \includegraphics[width, height, ...]{}
\setkeys{Gin}{width=\maxwidth,height=\maxheight,keepaspectratio}
% Set default figure placement to htbp
\makeatletter
\def\fps@figure{htbp}
\makeatother
\setlength{\emergencystretch}{3em} % prevent overfull lines
\providecommand{\tightlist}{%
  \setlength{\itemsep}{0pt}\setlength{\parskip}{0pt}}
\setcounter{secnumdepth}{5}
\newlength{\cslhangindent}
\setlength{\cslhangindent}{1.5em}
\newlength{\csllabelwidth}
\setlength{\csllabelwidth}{3em}
\newlength{\cslentryspacingunit} % times entry-spacing
\setlength{\cslentryspacingunit}{\parskip}
\newenvironment{CSLReferences}[2] % #1 hanging-ident, #2 entry spacing
 {% don't indent paragraphs
  \setlength{\parindent}{0pt}
  % turn on hanging indent if param 1 is 1
  \ifodd #1
  \let\oldpar\par
  \def\par{\hangindent=\cslhangindent\oldpar}
  \fi
  % set entry spacing
  \setlength{\parskip}{#2\cslentryspacingunit}
 }%
 {}
\usepackage{calc}
\newcommand{\CSLBlock}[1]{#1\hfill\break}
\newcommand{\CSLLeftMargin}[1]{\parbox[t]{\csllabelwidth}{#1}}
\newcommand{\CSLRightInline}[1]{\parbox[t]{\linewidth - \csllabelwidth}{#1}\break}
\newcommand{\CSLIndent}[1]{\hspace{\cslhangindent}#1}
\ifLuaTeX
  \usepackage{selnolig}  % disable illegal ligatures
\fi
\IfFileExists{bookmark.sty}{\usepackage{bookmark}}{\usepackage{hyperref}}
\IfFileExists{xurl.sty}{\usepackage{xurl}}{} % add URL line breaks if available
\urlstyle{same} % disable monospaced font for URLs
\hypersetup{
  hidelinks,
  pdfcreator={LaTeX via pandoc}}

\author{}
\date{\vspace{-2.5em}}

\begin{document}

The structure of all periodic matrices used in the two different scenarios are listed below. All numbers are female-only. Each theoretical matrix for a sub-annual period is followed by the set of matrices used in that sub-annual period. The abbreviate row and column names are:\\
- s\_t: seed at the top stratum (0 - 2 cm),\\
- s\_b: seed at the bottom stratum (2 - 20 cm),\\
- p\_co\_1 through p\_co\_6: plant cohort 1 through 6.

\hypertarget{published-literature-data}{%
\paragraph*{Published literature data}\label{published-literature-data}}
\addcontentsline{toc}{paragraph}{Published literature data}

\hypertarget{pre-planting-tillage-induced-vertical-redistribution-of-seeds}{%
\subparagraph*{Pre-planting tillage induced vertical redistribution of seeds}\label{pre-planting-tillage-induced-vertical-redistribution-of-seeds}}
\addcontentsline{toc}{subparagraph}{Pre-planting tillage induced vertical redistribution of seeds}

The only non-zeroes section of the pre-planting tillage induced vertical redistribution of seeds is \(M_s\). \(M_s\)'s were resized from the raw data of Seed Chaser (Spokas et al., 2007), a simulation program that estimates vertical seed movement after various types of tillage: the proportion of seeds staying at its original soil stratum, \(t_{11,s}\) and \(t_{22,s}\), or move to another stratum, \(t_{12,s}\) and \(t_{21,s}\). The original matrices in Spokas et al. (2007) were resized to 2 x 2 by summing over all the elements within each of the four sections, i.e., top left 2 x 2, bottom left 18 x 2, top right 2 x 18, and 18x18, and divide each of the i x 2 summations by the summation of the 20 x 2 left section, and each of the i x 18 summations by the summation of the 20 x 2 right section.

No-till is represented by an identical matrix, \(I\), after Cousens and Moss (1990). A field cultivator was applied before planting corn (C2, C3, and C4), soybean (S2, S3, and S4), and oat (O3 and O4). No tillage was applied before alfalfa (A4) because alfalfa that was intercropped with oat in the 4-year rotation (O4) was kept overwinter and grown as a sole crop in the following year.

\[
B_{t(s)} = \left[\begin{array}
{rr|rrrrr} 
t_{11,s} & t_{21,s} & 0 & ... & ... & ... & ... & 0\\
t_{12,s} & t_{22,s} & 0 & ... & ... & ... & ... & 0\\
\hline    
0 & 0 & 0 & ... & ... & ... & ... &0\\
... & ... & ... & ... & ... & ... & ... & ...\\ 
... & ... & ... & ... & ... & ... & ... & ...\\ 
0 & 0 & 0 & ... & ... & ... & ... & 0\\
\end{array}\right]
\]

The same pre-planting tillage regimes were applied in 2018 and 2019. The list of pre-planting tillage matrices is available at \url{https://github.com/hnguyen19/matrix-prospective/blob/master/2-Data/Clean/mean-pre-planting-tillage.RData}.

\hypertarget{in-season-survival-of-seeds-and-seedlings}{%
\subparagraph*{In-season survival of seeds and seedlings}\label{in-season-survival-of-seeds-and-seedlings}}
\addcontentsline{toc}{subparagraph}{In-season survival of seeds and seedlings}

The matrix \(B_s\) is comprised of seed survival rates at the \(M_s\) and plant survival rates at the \(M_p\) sections, respectively.

\[
B_s = \left[\begin{array}
{rr|rrrrr} 
s_{s_1}  & 0 & 0 & ... & ... & ... & ... & 0\\
0 & s_{s_2}  & 0 & ... & ... & ... & ... & 0\\
\hline             
0 & 0 & s_{p_1} & ... & ... & ...&...  & 0\\
... & ... & 0 & s_{p_2} & ... & ... & ... & 0\\ 
... & ... & ... & ... & ... & ... & ... & 0\\ 
0 & 0 & 0 & ... & ... & ...& 0 & s_{p_6}\\
\end{array}\right]
\]

The \(M_s\) section's diagonal (\(s_{s_1}\) and \(s_{s_2}\)) were filled with survival rates adapted from equations \(o_1 = y = 98.9 e^{-0.068x} \%\) and \(o_2 = y = 99.5 e^{-0.05x} \%\) (Figures 1 and 3, Sosnoskie et al., 2013) for the top and bottom layers. The values of x was assigned at 6 months for all crop environments. We settled at 6 months despite the complexity in tillage timing and method, light and humidity conditions, and granivores' activities at individual crop environments for simplicity. In reality, the burial length can interact with any crop management activity and deliver different germination and emergence results.

The empirically measured data for seedling survival were deemed unrealistically (Appendix) low as compared to the literature, so Nordby and Hartzler (2004)'s results were used for corn and Hartzler et al. (2004)'s results were used for soybean crop environments. The seedling survival rates by cohort (\(s_{p_k}, k = \{1,...,6\}\)) were assigned such that the earlier cohorts had lower survival rate in the oat crop environment; and those in the alfalfa crop environment were evenly low in all cohorts. These estimated numbers were based on a suggestion that cool-season crop environments can inhibit warm-season weed species growth (Nguyen and Liebman, 2022b and citations given there).

The same summer survival rates were used in 2018 and 2019. The list of summer seed survival rate matrices is available at \url{https://github.com/hnguyen19/matrix-prospective/blob/master/2-Data/Clean/mean-summer-seed-survival-Sosnoskie.RData} and the list of seedling survival rate to maturity is available at \url{https://github.com/hnguyen19/matrix-prospective/blob/master/2-Data/Clean/mean-summer-seedling-survival-Hartzler.RData}.

\hypertarget{post-harvest-tillage-induced-vertical-redistribution-of-seeds-post-harvest-tillage}{%
\subparagraph*{Post-harvest tillage induced vertical redistribution of seeds post-harvest tillage}\label{post-harvest-tillage-induced-vertical-redistribution-of-seeds-post-harvest-tillage}}
\addcontentsline{toc}{subparagraph}{Post-harvest tillage induced vertical redistribution of seeds post-harvest tillage}

The compilation of \(B_{t(f)}\) was the similar to that of \(B_{t(s)}\). Chisel plowing was applied after corn was harvested in the C2, C3, and C4 treatments, no-till was applied after harvests in the S2, S3, S4, and O4 treatments, and moldboard plowing was applied at the end of the O3 and A4 phases.

\[
B_{t(f)} = \left[\begin{array}
{rr|rrrrr} 
t_{11,f} & t_{21,f} & 0 & ... & ... & ... & ... & 0\\
t_{12,f} & t_{22,f} & 0 & ... & ... & ... & ... & 0\\
\hline    
0 & 0 & 0 & ... & ... & ... & ... & 0\\
... & ... & ... & ... & ... & ... & ... & ...\\ 
... & ... & ... & ... & ... & ... & ... & ...\\ 
0 & 0 & 0 & ... & ... & ... & ... & 0\\
\end{array}\right]
\]

The same post-harvest tillage regimes were applied in 2018 and 2019. The list of pre-planting tillage matrices is available at \url{https://github.com/hnguyen19/matrix-prospective/blob/master/2-Data/Clean/mean-post-harvest-tillage.RData}

\hypertarget{overwinter-survival}{%
\subparagraph*{Overwinter survival}\label{overwinter-survival}}
\addcontentsline{toc}{subparagraph}{Overwinter survival}

The compilation of matrix \(B_o\) was similar to that of \(B_s\), using equations \(o_1 = y = 98.9 e^{-0.068x} \%\) and \(o_2 = y = 99.5 e^{-0.05x} \%\) (Figures 1 and 3, Sosnoskie et al., 2013).

\[
\mathbf{B_o} = \left[\begin{array}
{rr|rrrrr} 
o_{11} & 0 & 0 & ... & ... & ... & ... & 0\\
0  & o_{22} & 0 & ... & ... & ... & ... & 0\\
\hline    
0 & 0 & 0 & ... & ... & ... & ... & 0 \\
... & ... & ... & ... & ... & ... & ... & ...\\ 
... & ... & ... & ... & ... & ... & ... & ...\\ 
0 & 0 & 0 & ... & ... & ... & ... & 0\\
\end{array}\right]
\]

The same overwinter survival rates were used in 2018 and 2019. Some zero values were due to rounding. The list of overwinter seed survival matrices is available at \url{https://github.com/hnguyen19/matrix-prospective/blob/master/2-Data/Clean/mean-winter-seed-survival-Sosnoskie.RData}

\hypertarget{empirically-measured-data}{%
\paragraph*{Empirically measured data}\label{empirically-measured-data}}
\addcontentsline{toc}{paragraph}{Empirically measured data}

\hypertarget{seedling-recruitment}{%
\subparagraph*{Seedling recruitment}\label{seedling-recruitment}}
\addcontentsline{toc}{subparagraph}{Seedling recruitment}

The emergence proportions calculated from step 5 here are positioned on the first column of block \(M_{s,p}\) in matrix \(B_g\). \(1 - sum_{k=1}^6 g_k\) represents the proportion of non-emerging seeds.

\[
\mathbf{B_g}=\left[\begin{array}
{rr|rrrrr} 
d = 1-\sum_{k=1}^6 g_k & 0 & 0 & ... & ... & ... & ... & 0\\  
0 & 1 & 0 & ... & ... & ... & ... & 0\\  
\hline             
g_1 & 0 & 0 & ... & ... & ... & ... & 0\\  
... & ... & ... & ... & ... & ... & ... & ...\\  
g_6 & 0 & 0 & ... & ... & ... & ... & 0\\  
\end{array}\right]
\]

The proportion of seedling emergence from the top 0-2 cm soil seedbank stratum in each crop identity crossed with corn weed management regime was calculated with the following steps:

1 - Estimate the 0-2 cm and 2-20 cm seedbank densities with the soil seedbank samples collected before post-harvest tillage. A seed column at a particular sub-annual period is comprised of the 0-2 cm and 2-20 cm soil stratum seed densities, \(N_h = [s_t, s_b]\).

From steps 2 through 4, the seed column in sub-period h, \(N_h\), was transitioned from one period to the next with the general matrix multiplication of \(N_{h+1} = B_hN_h\) by Caswell (2001).

2 - Estimate post-harvest tillage induced seed vertical redistribution with resized Seed Chaser (Spokas et al., 2007) chisel and moldboard plowing matrices, as detailed in the \emph{Post-harvest tillage induced seed vertical movement}, to yield \(N_{t(f)}\)

3 - Adapt overwinter survival rates as previously explain in he \emph{Overwinter survival section} and apply it on \(N_{t(f)}\) to yield \(N_o\). Corn weed management did not affect waterhemp's first cohort emergence in the same crop environment (Appendix), so the same value of \(x_s\) was used for the same crop identity.

4 - Estimate pre-planting tillage induced seed vertical redistribution with resized Seed Chaser (Spokas et al., 2007) field cultivator matrix, similar to step 2 to yield \(N_{t(s)}\).

5 - Divide the seedling density in each cohort, \(l_k\), by \(s_{11,f}\), the top soil stratum seed density to yield \(g_k\).

The same emergence rates were used in 2018 and 2019. Some zero values in the first column were due to rounding. The list of seedling recruitment matrices is available at \url{https://github.com/hnguyen19/matrix-prospective/blob/master/2-Data/Clean/mean-emergence-prop-adjusted.RData}

\hypertarget{plant-fecundity}{%
\subparagraph*{Plant fecundity}\label{plant-fecundity}}
\addcontentsline{toc}{subparagraph}{Plant fecundity}

The plant fecundity matrix (\(B_f\)) had the \(M_s\) block's diagonal filled with 1's and the first row of the \(M_{p,s}\) filled with \(f_k, k = \{1,...,6\}\). The 1's in the \(M_s\) block's diagonal are placeholders to carry the product from the previous matrices over.

\[
\mathbf{B_f} = \left[\begin{array}
{rr|rrrrr} 
1 & 0 & f_1 & f_2 & ... & ... & ... & f_6\\
0 & 1 & 0 & ... & ... & ... & ... & 0\\
\hline   
0 & 0 & 0 & ... & ... & ... & ... & 0\\
... & ...& ... & ... &...&...&...& 0\\
... & ... & ... & ... & ... & ... & ... & ...\\ 
0 & 0 & 0 & ... & ... & ... & 0 & 0\\
\end{array}\right]
\]

Two scenarios of plant fecundity were used. In scenario 1, plant fecundity (\(f_k, k =\{1,...,6\}\)) in each crop identity crossed with corn weed management was estimated from plant aboveground mass using eighteen equations from Nguyen and Liebman (2022a). In scenario 2, the plants were partitioned into six size-based bins and their fecundity was summarized as \(f_k, k =\{1,...,6\}\) and filled in their relevant positions in the \(B_f\) matrix by partitioning. Both practices in scenarios 1 and 2 were based on the assumption that plant size and fecundity decreased as emergence delayed (Hartzler et al., 2004; Nordby and Hartzler, 2004).

Scenario 1: High control efficacy

The list of cohort-averaged fecundity matrices under high control efficacy is available at \url{https://github.com/hnguyen19/matrix-prospective/blob/master/2-Data/Clean/mean-fecundity-19-cohort.RData}

Scenario 2: Low control efficacy

The list of cohort-averaged fecundity matrices under low control efficacy is available at \url{https://github.com/hnguyen19/matrix-prospective/blob/master/2-Data/Clean/mean-fecundity-18-cohort.RData}

\hypertarget{refs}{}
\begin{CSLReferences}{1}{0}
\leavevmode\vadjust pre{\hypertarget{ref-caswellMatrixPopulationModels2001}{}}%
Caswell, H. (2001). \emph{Matrix population models: Construction, analysis, and interpretation} (Second). {Sunderland, Mass. : Sinauer Associates}.

\leavevmode\vadjust pre{\hypertarget{ref-cousensModelEffectsCultivation1990}{}}%
Cousens, R., and Moss, S. R. (1990). A model of the effects of cultivation on the vertical distribution of weed seeds within the soil. \emph{Weed Research}, \emph{30}(1), 61--70. \url{https://doi.org/d824tt}

\leavevmode\vadjust pre{\hypertarget{ref-hartzlerEffectCommonWaterhemp2004}{}}%
Hartzler, R. G., Battles, B. A., and Nordby, D. (2004). Effect of common waterhemp ({\emph{Amaranthus rudis}}) emergence date on growth and fecundity in soybean. \emph{Weed Science}, \emph{52}(2), 242--245. \url{https://doi.org/cmhpxk}

\leavevmode\vadjust pre{\hypertarget{ref-nguyenImpactCroppingSystem2022}{}}%
Nguyen, H. T. X., and Liebman, M. (2022a). Impact of cropping system diversification on vegetative and reproductive characteristics of waterhemp ({\emph{A. tuberculatus}}). \emph{Frontiers in Agronomy}, \emph{4}. \url{https://doi.org/gpsrmj}

\leavevmode\vadjust pre{\hypertarget{ref-nguyenWeedCommunityComposition2022}{}}%
Nguyen, H. T. X., and Liebman, M. (2022b). Weed community composition in simple and more diverse cropping systems. \emph{Front. Agron.} \url{https://doi.org/gpsrmk}

\leavevmode\vadjust pre{\hypertarget{ref-nordbyInfluenceCornCommon2004}{}}%
Nordby, D. E., and Hartzler, R. G. (2004). Influence of corn on common waterhemp ({\emph{Amaranthus rudis}}) growth and fecundity. \emph{Weed Science}, \emph{52}(2), 255--259. \url{https://doi.org/10.1614/WS-03-060R}

\leavevmode\vadjust pre{\hypertarget{ref-sosnoskieGlyphosateResistanceDoes2013}{}}%
Sosnoskie, L. M., Webster, T. M., and Culpepper, A. S. (2013). Glyphosate resistance does not affect {Palmer} amaranth ({Amaranthus} palmeri) seedbank longevity. \emph{Weed Science}, \emph{61}(2), 283--288. \url{https://doi.org/f4vgfs}

\leavevmode\vadjust pre{\hypertarget{ref-spokasSeedChaserVerticalSoil2007}{}}%
Spokas, K., Forcella, F., Archer, D., and Reicosky, D. (2007). {SeedChaser}: {Vertical} soil tillage distribution model. \emph{Computers and Electronics in Agriculture}, \emph{57}(1), 62--73. \url{https://doi.org/dzh845}

\end{CSLReferences}

\end{document}
